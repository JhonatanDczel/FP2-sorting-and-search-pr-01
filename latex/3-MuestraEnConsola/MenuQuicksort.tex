\subsection{Ejecución de QuickSort}
\begin{itemize}
  \item A continuacion veremos la ejecucion del programa en la terminal:
  \item \textbf{Accedemos por medio del menú}
  \end{itemize}
   	\begin{lstlisting}[language=bash,caption={Compilación y ejecución del código}][H]
    javac Comparation.java
    java Comparation
=======Menu Principal========
1. Ordenamiento por Inserción
2. Ordenamiento por Quicksort
3. Búsqueda Binaria Recursiva
4. Búsqueda Binaria Iterativa
5. Comparar Rendimientos
6. Salir
Seleccione una opción: 1

=======Ordenar por:========
1. cui
2. email
3. name
4. apellido paterno
5. apellido materno
6. F. nacimiento
7. Genero
8. Status
1
CUI: 10654321 Email: rmendozarodriguez@unsa.edu.pe Nombre: Raul	 A. Pat: Mendoza A. Mat: Rodriguez Fecha de Nacimiento: 1992-08-07 Genero: 0 Estado: 1
CUI: 11789012 Email: jperezgomez@unsa.edu.pe Nombre: Javier	 A. Pat: Perez A. Mat: Gomez Fecha de Nacimiento: 1990-12-30 Genero: 0 Estado: 1
CUI: 12876543 Email: cmendozarivera@unsa.edu.pe Nombre: Carlos	 A. Pat: Mendoza A. Mat: Rivera Fecha de Nacimiento: 1991-03-05 Genero: 0 Estado: 1

  \end{lstlisting}
\end{itemize}




\begin{comment}
%COMMIT----------------------------------------------
  \begin{itemize}
    \item \textbf{Hash: aquí debe haber consola}
    \item
  \end{itemize}
   	\begin{lstlisting}[language=bash,caption={Compilación y ejecución del código}][H]
  \end{lstlisting}
\end{comment}

