\subsection{Parte grafica}
%Aqui desarrollas la explicacion de tu codigo
Necesitamos una graficadora, que interprete el comportamiento de los algoritmos de ordenamiento, en un tiempo y muestra determinados\dots
\begin{itemize}
  \item Para hacerlo usaremos GNUPlot
  \item Necesitaremos un Script que muestre por pantalla la grafica de los algoritmos
\end{itemize}
\subsubsection{Script de bash para GNUPlot}

\begin{lstlisting}[language=bash, caption={Script de GNUPlot}, numbers=left, firstnumber=1][H]

#Definimos el nombre por defecto en caso de no haber ingresado un nombre en el segundo parametro
if [ "$outputFile" == ".png" ]; then
	outputFile="Grafica.png"
fi

#Ahora definimos los nombres de los archivos
grafico1=$(echo "$2" | cut -c 18- | rev | cut -c 5- | rev)
grafico2=$(echo "$3" | cut -c 18- | rev | cut -c 5- | rev)
#Definimos la linea que se usara para la salida del plot
plotLine="plot \"$2\" w lp lw 3 lc rgb \"#62aeef\" pt 6 title \"$grafico1\""
if [[ $3 != "" ]]; then
	plotLine="$plotLine, \"$3\" w lp lw 3 lc rgb \"#ddf056\" pt 6 t \"$grafico2\""
fi

# aqui se ejecuta Gnuplot para generar el gráfico y exportarlo
gnuplot <<-EOF
	# Cambiamos el output de la terminal a png
	set terminal pngcairo enhanced
	set output "./graphics/imgs/$outputFile"

	set title "$1" textcolor rgb "white" font "helvetica,12"
	set key left box textcolor "white" font "helvetica,12"
	set object 1 rectangle from screen 0,0 to screen 1,1 behind noclip fc rgb "#32363d" fillstyle solid 1.0
	set border lc rgb "white" lw 2
	set xlabel "Data count" font "helvetica,12" textcolor "white"
	set ylabel "Time (ns)" font "helvetica,12" textcolor "white"
	set tics font "helvetica,11" textcolor rgb "#faf95d"

	# Se crea la grafica
	$plotLine
EOF

xdg-open "./graphics/imgs/$outputFile"
echo "El gráfico se ha generado como $outputFile"  

\end{lstlisting}
\begin{itemize}
  \item Este script recibe dos parametros:
  \item Nombre para la imagen del resultado
  \item Archivo .dat que se graficara 

\end{itemize}
\subsubsection{Menu}
\begin{itemize}
  \item Necesitamos un menu para pedir entradas al usuario, nuestro se vera de la siguiente manera:
\end{itemize}

\begin{lstlisting}[language=java, caption={Menu}, numbers=left, firstnumber=1][H]
      System.out.println("\n=======Menu Principal========");
      System.out.println("1. Ordenamiento por Inserción");
      System.out.println("2. Ordenamiento por Quicksort");
      System.out.println("3. Búsqueda Binaria Recursiva");
      System.out.println("4. Búsqueda Binaria Iterativa");
      System.out.println("5. Comparar Rendimientos");
      System.out.println("6. Salir");
      System.out.print("Seleccione una opción: ");

\end{lstlisting}
\begin{itemize}
  \item Adicionalmente, tendremos que mostrarle las opciones de ordenamiento para cada uno de los algoritmos:
\end{itemize} 
\begin{lstlisting}[language=bash, caption={Menu}, numbers=left, firstnumber=1][H]
          System.out.println("\n=======Ordenar por:========");
          System.out.println("1. cui");
          System.out.println("2. email");
          System.out.println("3. name");
          System.out.println("4. apellido paterno");
          System.out.println("5. apellido materno");
          System.out.println("6. F. nacimiento");
          System.out.println("7. Genero");
          System.out.println("8. Status");
\end{lstlisting}
\begin{itemize}
  \item Con esto pediremos dos entradas al usuario, que seran almacenadas en dos variables de tipo entero 

\end{itemize}

La estructura de ejecucion de nuestro menu, consta de switchs anidados, el primero incluye los elementos del primer menu, y el segundo incluye los elementos del segundo menu



\subsubsection{Clase Test}
\begin{itemize}
  \item Ahora necesitamos una clase Test, que corra un algoritmo con casos de prueba
  \item Tenemos dos metodos principales, runAlgorithm y compareAlgorith
  \item La clase tiene los siguientes atributos:
  \begin{itemize}
    \item String name, se refiere al algoritmo que se esta ejecutando
    \item Reader.Student[] sujetosPrueba, es la copia de array para hacer los ordenamientos (Importante: Para evitar el conflicto de pasar objeto por referencia en lugar de valor, cada llamada a Test copiara un nuevo arreglo identico al original para asegurarnos de tener siempre un arreglo desordenado)
  \end{itemize}
\end{itemize}
\subsubsection{selectUser: Metodo auxiliar}
\begin{itemize}
  \item Para poder ejecutar el algoritmo correcto segun la entrada del usuario, necesitaremos un metodo que tome la entrada del usuario y corra el algoritmo correcto
  \item Las entradas para el metodo son, son dos enteros representando la eleccion 1 y eleccion 2 del usuario, y un arreglo de estudiantes para ordenar
\end{itemize}


\begin{lstlisting}[language=java, caption={Metodo auxiliar}, numbers=left, firstnumber=1][H]

 static public void selectUser(int o1, int o2, Reader.Student[] muestra){
      int option = o1 * 10 + o2;
      switch (option) {
        case 11:
          QuickSort.cui(muestra, 0, muestra.length - 1);
          name = "CUI";
          break;
        case 12:
          QuickSort.email(muestra, 0, muestra.length - 1);
          name = "Email";
          break;
        case 13:
          QuickSort.name(muestra, 0, muestra.length - 1);
          name = "Nombre";
          break;
        case 14:
          QuickSort.lastNameF(muestra, 0, muestra.length - 1);
          name = "Apellido Paterno";
          break;
          ...
\end{lstlisting}
\subsubsection{Metodo runAlgorithm}
\begin{itemize}
  \item El metodo se encarga de hacer un test a un determinado algoritmo
\end{itemize}
\begin{lstlisting}[language=java, caption={Testear algoritmo}, numbers=left, firstnumber=1][H]
    static public void runAlgorithm(int o1, int o2){
      String data = "";
      for(int i = 0; i < 100; i++){
        if(i == 0)
          continue;
        Reader.Student[] muestra = Arrays.copyOf(sujetosPrueba, testCases[i]);
        long startTime = System.nanoTime();
        selectUser(o1, o2, muestra);
        long endTime = System.nanoTime();
        String time = Long.toString(endTime - startTime);
        data = data + "\n" + String.valueOf(testCases[i]) + "\t" + time;
      }
      try (BufferedWriter bw = new BufferedWriter(new FileWriter("./graphics/input/data.dat"))) {
        bw.write(data);
      } catch (IOException e) {
        e.printStackTrace();
      }
      String comando = "./graphics/graficar.sh \"" + name + "\" ./graphics/input/data.dat";
      try {
          ProcessBuilder builder = new ProcessBuilder();
          builder.command("sh", "-c", comando);
          Process proceso = builder.start();
      } catch (IOException e) {
          e.printStackTrace();
      }
    }
\end{lstlisting}


\begin{itemize}
  \item Como vemos, se recorre un ciclo for 100 veces, y en cada ciclo se llama a el metodo selectUser (que ejecutara el algoritmo correcto segun lo que el usuario ingrese) y guargara los datos de cada ejecucion en el String data
  \item Posteriormente se escribe un archivo llamado data.dat que contiene los datos de ejecucion del algoritmo 
  \item Con esto, se procede a ejecutar por consola el comando que llamara a nuestro scritp anteriormente mostrado, para generar una grafica por pantalla
\end{itemize}
\subsubsection{Metodo compareAlgorith}
\begin{itemize}
  \item Para esto, necesitaremos hacer una doble implementacion del anterior metodo,
  \item Escribiremos dos archivos de dos algoritmos respectivos y luego ejecutaremos el script con esos dos parametros

\end{itemize}
\begin{lstlisting}[language=java, caption={Testear algoritmo}, numbers=left, firstnumber=1][H]
    static public void compareAlgorithm(){
      String data1 = "";
      String data2 = "";
      for(int i = 0; i < 100; i++){
        if(i == 0)
          continue;
        Reader.Student[] muestra1 = Arrays.copyOf(sujetosPrueba, testCases[i]);
        Reader.Student[] muestra2 = Arrays.copyOf(muestra1, muestra1.length);
        long startTime1 = System.nanoTime();
        selectUser(1, 6, muestra1);
        long endTime1 = System.nanoTime();
        String time1 = Long.toString(endTime1 - startTime1);
        data1 = data1 + "\n" + String.valueOf(testCases[i]) + "\t" + time1;

        long startTime2 = System.nanoTime();
        selectUser(2, 6, muestra2);
        long endTime2 = System.nanoTime();
        String time2 = Long.toString(endTime2 - startTime2);
        data2 = data2 + "\n" + String.valueOf(testCases[i]) + "\t" + time2;
      }
      String file1 = "./graphics/input/InsertionSort.dat";
      try (BufferedWriter bw = new BufferedWriter(new FileWriter(file1))) {
        bw.write(data1);
      } catch (IOException e) {
        e.printStackTrace();
      }

      String file2 = "./graphics/input/QuickSort.dat";
      try (BufferedWriter bw = new BufferedWriter(new FileWriter(file2))) {
        bw.write(data2);
      } catch (IOException e) {
        e.printStackTrace();
      }

      String comando = "./graphics/graficar.sh \"Comparacion de algoritmos\" " + file1 + " " + file2;
      try {
          ProcessBuilder builder = new ProcessBuilder();
          builder.command("sh", "-c", comando);
          Process proceso = builder.start();
      } catch (IOException e) {
          e.printStackTrace();
      }
    }

\end{lstlisting}
\begin{itemize}
  \item Luego de llamar a una comparacion, tendremos una grafica por pantalla que nos mostrara el comportamiento de los dos algoritmos de ordenamiento
\end{itemize}
\subsubsection



\subsubsection{Commits principales}
%Aqui muestras los commits mas relevantes

\begin{lstlisting}[language=java, caption={Testear algoritmo}, numbers=left, firstnumber=1][H]
commit 36c13d21d090dfff3745e873cd488fb5f6bf99b5
Author: JhonatanDczel <jariasq@unsa.edu.pe>
Date:   Wed Oct 4 13:11:18 2023 -0500

    Comparation, corrigiendo errores
\end{lstlisting}
\begin{itemize}
  \item Previamente se subio una version de la Clase Test, que ejecutaba los casos de prueba de un determinado algoritmo
  \item En esta version, se corrigieron errores de sintaxis y de ejecucion
\end{itemize}
