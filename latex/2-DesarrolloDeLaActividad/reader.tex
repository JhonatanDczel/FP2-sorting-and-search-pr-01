\subsection{Lectura y almacenamiento de datos}

\subsubsection{Descripción}

La clase \texttt{Reader} se encarga de leer los datos de un archivo CSV y almacenarlos en un arreglo de objetos \texttt{Student}. Los pasos principales son:

\begin{itemize}
\item Abrir el archivo \texttt{data.csv} 
\item Leer linea por linea con \texttt{BufferedReader}
\item Separar cada linea por comas para obtener los campos
\item Crear un nuevo objeto \texttt{Student}
\item Setear los campos leídos en el objeto \texttt{Student}
\item Agregar el objeto \texttt{Student} a un \texttt{ArrayList}
\item Convertir el \texttt{ArrayList} a un arreglo al final
\end{itemize}

\subsubsection{Código relevante}

\begin{lstlisting}[language=Java,caption=Método para leer archivo CSV,]
private void readDataFile(){

  BufferedReader reader = null;

  String line = "";

  String[] parts;

  try {

    FileReader fileReader = new FileReader("./reader/data.csv");

    reader = new BufferedReader(fileReader);

    while ((line = reader.readLine()) != null) {

      parts = line.split(",");
      
      //... creación y seteo de Student

    }

  }catch (Exception e){

    e.printStackTrace();

  }finally{

    //...cerrar reader

  }

}
\end{lstlisting}

\begin{lstlisting}[language=Java,caption=Conversión de ArrayList a arreglo]
classmates = students.toArray(new Student[students.size()]);
\end{lstlisting}

\subsubsection{Explicación}

Se utiliza \texttt{BufferedReader} para leer el archivo línea por línea de forma eficiente. Cada línea se divide en las partes correspondientes a cada campo, delimitadas por comas. Luego se crea un objeto \texttt{Student}, se setean sus campos y se agrega al \texttt{ArrayList}. Al final se convierte el \texttt{ArrayList} a arreglo para retornarlo.