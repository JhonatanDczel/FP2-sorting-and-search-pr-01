
	
	\vspace*{10pt}
	
	\begin{center}	
		\fontsize{17}{17} \textbf{ Informe \itemPracticeNumber}	
	\end{center}
	\begin{minipage}{\textwidth}
	\cen
	\textbf{\Large \itemTheme}
	\vspace*{0.5cm}
	\end{minipage}

	\begin{flushright}
		\begin{tabular}{|M{2.5cm}|N|}
			\hline 
			\rowcolor{tablebackground}
			\color{white} \textbf{Nota}  \\
			\hline 
			     \\[30pt]
			\hline 			
		\end{tabular}
	\end{flushright}	

	\begin{table}[H]
		\begin{tabular}{|x{5.5cm}|x{4.5cm}|x{4.2cm}|}
			\hline 
			\rowcolor{tablebackground}
			\color{white} \textbf{Estudiantes} & \color{white}\textbf{Escuela}  & \color{white}\textbf{Asignatura}   \\
			\hline 
			{\itemStudent \par \itemEmail} & \itemSchool & {\itemCourse \par Semestre: \itemSemester \par Código: \itemCourseCode}     \\
			\hline 			
		\end{tabular}
	\end{table}		
	
	\begin{table}[H]
		\begin{tabular}{|x{4.7cm}|x{4.8cm}|x{4.8cm}|}
			\hline 
			\rowcolor{tablebackground}
			\color{white}\textbf{Teoría} & \color{white}\textbf{Tema}  & \color{white}\textbf{Duración}   \\
			\hline 
			\itemPracticeNumber & \itemTheme & 04 horas   \\
			\hline 
		\end{tabular}
	\end{table}
	
	\begin{table}[H]
		\begin{tabular}{|x{4.7cm}|x{4.8cm}|x{4.8cm}|}
			\hline 
			\rowcolor{tablebackground}
			\color{white}\textbf{Semestre académico} & \color{white}\textbf{Fecha de inicio}  & \color{white}\textbf{Fecha de entrega}   \\
			\hline 
			\itemAcademic & \itemInput &  \itemOutput  \\
			\hline 
		\end{tabular}
	\end{table}

	\section{Actividad}
	\begin{itemize}		
	  \item Elaborar un proyecto utilizando git. donde se elabore un sistema para ingresar datos de alumnos universitarios. (Clase Student)
	  \item El sistema debe almacenar los estudiantes en un Array. (Considerar leer archivos CSV).
	  \item Implemente el algoritmo de ordenamiento por Inserción(Iterativo-Cuadrático) para ordenar el arreglo de estudiantes por diferentes parámetros. Ejemplo: Por apellido, paterno. Descubra cuál es el tiempo que se demora en las ejecuciones.
	  \item Explique cualquier otro algoritmo de ordenamiento de complejidad logarítmica. e implemente el ordenamiento utilizando los mismo parámetros anteriores.
	  \item Grafique los resultados de las simulaciones realizadas considerando como unidad de medida los nanosegundos. Desde n=1 alumno hasta n=N alumnos.
	  \item Luego, para el arreglo ordenado implemente el algoritmo de búsqueda binaria iterativo/recursivo y grafique los resultados de sus simulaciones.
 

	\end{itemize}
		
	\section{Equipos, materiales y temas utilizados}
	\begin{itemize}
		\item Sistema Operativo ArchCraft GNU Linux 64 bits Kernell, Sistema Operativo Ubuntu GNU Linux 22.04, Sistema operativo windows 11
		\item NeoVim, vs code
		\item OpenJDK 64-Bit 20.0.1
		\item Gnuplot 5.4.9
		\item Git 2.42.0
		\item Cuenta en GitHub con el correo institucional.
		\item Programación Orientada a Objetos.
		\item Algoritmos de ordenamiento y búsqueda.
	\end{itemize}

	\section{URL de Repositorio Github}
	\begin{itemize}
            \item URL para acceder a la \itemPracticeNumber{} en el Repositorio GitHub.
            \item \itemUrl
	\end{itemize}





